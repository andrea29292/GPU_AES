\chapter{Advanced Encryption Standard (AES)}
Sviluppato dai due crittografi belgi Joan Daemen e Vincent Rijmen l'Advanced Encryption Standard (AES), conosciuto anche come Rijndael, di cui più propriamente è una specifica implementazione, è un algoritmo di cifratura a blocchi utilizzato come standard dal governo degli Stati Uniti d'America. 

\paragraph{}Data la sua sicurezza e le sue specifiche pubbliche si presume che in un prossimo futuro venga utilizzato in tutto il mondo come è successo al suo predecessore, il Data Encryption Standard (DES) che ha perso poi efficacia per vulnerabilità intrinseche. AES è stato adottato dalla National Institute of Standards and Technology (NIST) e dalla US FIPS PUB nel novembre del 2001 dopo 5 anni di studi, standardizzazioni e selezione finale tra i vari algoritmi proposti.

\section{Descrizione dell'algoritmo}
AES opera utilizzando matrici di 4x4 byte chiamate \textbf{stati}. Quando l'algoritmo ha blocchi di 128 bit in input, la matrice si stato ha 4 righe e 4 colonne; se il numero di blocchi in input diventa di 32 bit più lungo, viene aggiunta una colonna allo stato, e così via fino a 256 bit. In pratica, si divide il numero di bit del blocco in input per 32 e il quoziente specifica il numero di colonne.